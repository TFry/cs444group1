\documentclass{article}
\usepackage[utf8]{inputenc}
\usepackage{hyperref}
\usepackage[margin=0.75in]{geometry}
\begin{document}
\section{Plan to implement the necessary algorithm}
After doing some research, we found a basic, working version of SBD. Then, we just added a few small components. Besides using basic initialization and clean-up function provided by Linux Crypto, the only thing needed was to use provided functions to encrypt/decrypt one block at a time.

\section{Main Point of this Assignment}
The main point of this assignment was to implement a block device in our local linux 
kernel build that encrypts and decrypts data when it is used to write or read data. Our main
goal for this assignment was to find a pre-written block device driver, and learn how to 
build it and use it in our kernel to accomplish the assignment's requirements. On top of that 
we were supposed to use one of the provided cryptography files found in the kernel /crypto 
folder in order to properly encrypt and decrypt the I/O that went through the device.
\\\\
The main point of the concurrency assignment was to (obviously) work on our concurrency skills.
This one was different than the last two assignments due to the three different tasks all
having to work in unison. Even in the little book of semaphores this problem is listed in the
"non-classical problems", so it seems like we're straying away from the more traditional 
problems to ones that might show up in the "real world".
\section{How did you ensure your solution was correct? Testing details, for instance.}
We tested the concurrency by writing up a function that prints out the current list at every
step. This coupled with the print statements associated with each thread (that prints out 
whether an item was searched for, inserted, or deleted) paints a picture of what is happening
behind the scenes. You can see the list before a delete thread comes in and after it comes in
and you will see an element omitted from the second list, proving that it was deleted
successfully.

For the block device, we have a script that loads the device as a module, creates a filesystem, mounts it, and writes and reads to it to make sure everything works as intended.

\section{What did you learn?}
We learned how to correctly compile a .c file and produce a .ko file which acts as a module
in the kernel. This module is then mounted onto the kernel and can be immediately put to use.
Not only that, but we learned how to use the Linux kernel crypto API and couple that with our
device implementation to create a functional device driver that can encrypt and decrypt the
data that it processes.
\\\\
For the concurrency assignment, we learned how to make use of multiple semaphores at once
to orchestrate more than 2 moving parts. We learned how to take the blocking requirements
and translate those directly to the semaphores in the code and prevent any operations that
weren't allowed.

\section*{Version Control Log}
\begin{tabular}{ l | c | r }
  \hline	
 c2e835d &	heidiaclayton     &	Mon May 22 01:11:11 2017 -0700 \\
0b731ba &	heidiaclayton     &	Mon May 22 00:58:25 2017 -0700 \\
90a4fbc &	heidiaclayton     &	Mon May 22 00:36:13 2017 -0700 \\
a6e6511 &	heidiaclayton     &	Mon May 22 00:26:20 2017 -0700 \\
dad7b1f &	heidiaclayton     &	Mon May 22 00:23:25 2017 -0700 \\
6458b3d &	heidiaclayton     &	Sun May 21 23:14:45 2017 -0700 \\
bed66cb &	heidiaclayton     &	Sun May 21 23:13:49 2017 -0700 \\
114b31b &	heidiaclayton     &	Sun May 21 23:10:42 2017 -0700 \\
0d4993f &	heidiaclayton     &	Sun May 21 23:08:24 2017 -0700 \\
07ea9ad &	heidiaclayton     &	Sun May 21 03:39:58 2017 -0700 \\
b0f4cb1 &	heidiaclayton     &	Sun May 21 03:38:54 2017 -0700 \\
613c019 &	heidiaclayton     &	Sun May 21 02:57:24 2017 -0700 \\
ff2034b &	heidiaclayton     &	Sun May 21 02:53:57 2017 -0700 \\
72a45a8 &	TannerFry         &	Thu May 18 12:42:38 2017 -0700 \\
1aecc87 &	TannerFry         &	Thu May 18 12:40:02 2017 -0700 \\
558d0c5 &	heidiaclayton     &	Thu May 18 12:33:36 2017 -0700 \\
226cced &	davidteofilovic	  & Thu May 18 12:14:58 2017 -0700 \\
81f390e &	davidteofilovic   &	Thu May 18 12:12:49 2017 -0700 \\
4ed7d2f &	heidiaclayton     &	Thu May 18 10:41:54 2017 -0700 \\
91e79af &	heidiaclayton     &	Tue May 16 18:56:21 2017 -0700 \\
a5ead0c &	heidiaclayton     &	Tue May 16 11:24:48 2017 -0700 \\
1f5a668 &	heidiaclayton     &	Mon May 15 12:52:22 2017 -0700 \\
b296aa6 &	heidiaclayton     &	Sun May 14 22:28:48 2017 -0700 \\
a6c997b &	heidiaclayton     &	Thu May 11 11:05:36 2017 -0700 \\
70cab44 &	heidiaclayton     &	Thu May 11 11:02:49 2017 -0700 \\
be081b6 &	heidiaclayton     &	Thu May 11 10:28:08 2017 -0700 \\
390cc1f &	heidiaclayton     &	Thu May 11 10:16:13 2017 -0700 \\ 
   \hline  
\end{tabular}

\section*{Work Log}
\subsection*{David}
We started the assignment pretty early, which was good because we got a lot of the ground 
work out of the way and asked Kevin a few questions before ultimately meeting again. During
recitation Heidi gave us a great head start on the concurrency assignment. Later that week,
we met up again and I primarily worked on the concurrency. I remembered in class Kevin
mentioned that the little book of semaphores would be a big help, so I found a copy online 
and looked at the description of the problem in there. From there I stripped out our current
implementation, which was trying to use both semaphores and thread mutex locks, and just
rewrote the permission logic to just use semaphores. Using the head start Heidi gave us and
the pseudo-code from the little book of semaphores, I was able to write it in a way that
satisfied all the requirements.

\subsection*{Heidi}
During week 6, David and I met to start preliminary work on the assignment, which was mainly research. During week 7, I implemented a primary version of the concurrency that was barely functional. Then, David debugged it and made it work completely. Tanner worked on a primary version of the block device which I then worked with to create the final version which ended up being quite frustrating. Turns out, I flipped two parameters in the decrypt function. 

\subsection{Tanner}
We did a first run over of the assignment two weeks out to get started. 
From there we did a bunch of reading and research to figure out what to do.
The concurrency assignment we started on next, I made a rough working model of it before class.
Heidi had started on a more structure working code version in class and we ran with that one.
David and Heidi did most of the work on the concurrency while I started on the assignment.
I wrote a first pass at encryption block device which should have worked.
Heidi did some work with it and refactored it twice to get it working.
Finally I did some document work and polishing and turned everything in. 

\end{document}