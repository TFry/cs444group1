\documentclass{article}
\usepackage[utf8]{inputenc}
\usepackage{hyperref}
\usepackage[margin=0.75in]{geometry}
\begin{document}
\section{What was the main point of this assignment?}
The main point of this assignment was to get familiar with 2 popular algorithm implementations
for memory management in the Linux kernel. Once we learned about these algorithms we were given
the task to switch out the first-fit algorithm (SLOB) with the best fit algorithm (SLAB). This
assignment forced us to really think about the low level implementation of memory management 
inside the kernel in two parts. The first part was the actual swapping of the algorithms. This 
involves changing the method used to find a place to allocate a page. The second part is actual
proving that the algorithm we implemented was correctly implemented. In order to do this, the
assignment also forced us to create a system call that could retrieve the actual memory usage.
\section{How did we personally attack the problem}
\subsection{Tanner}
The first thing I did was some basic research into what the Slob was. 
Then into how it functions and what its purpose within the Kernel is.
Once I had a decent understanding of how the Slob functioned I started to create a concept in my mind of how a Slob Slab would work.
The main difference being that the Slob Slab must be able to do best fit placement rather than first fit.
This means to place each chunk of data in the best fitting available slot.
Basically we need a data structure to store the all the available pages and keep track of the current best fit.
\subsection{Heidi}
I was unable to make our first meeting to work on this when the concurrency was finished. During our second meeting we looked at references and tried to bang out a first go at our best-fit algorithm. It didn't work so well so I worked on it some more over the weekend and the course of the next week and I managed to figure out how to make a new system call. Thursday of week 10 was stressful as our program would not work during our entire meeting. It would hang and the kernel would not boot. I managed to fix it somehow later that night by just rewriting what I had previously implemented and I apparently fixed it. I wrote a short test as well. 

\subsection{David}
I started this assignment in the most logical way you could, I looked at the original SLOB 
implementation located in the /mm/ directory and tried to decompose its functionality function
by function. It became increasingly clear that the two functions that we would have to pay
attention to were \textit{slob\_page\_alloc} and  \textit{slob\_alloc}. Slob\_alloc, as the 
name suggests, is where the actual allocation of the pages is being done. This is where we
ended up after the first meeting. During our second meeting we found some previous 
implementations. We wrote what we thought represented the algorithm and how it was meant to be
implemented but the kernel got hung up every time we tried to boot it. The majority of our time
on this assignment was spent trying to debug this, and it was a very annoying process. 
Thankfully, because of Heidi, we persevered and got through the hang up and the kernel booted
up. Heidi in her own time wrote up a test script for this that showed the differences between
the two algorithms.


\section{Testing Method}
We implemented system calls that returned the free and used units and then ran a short test script that did some large memory allocations. The ratio of used over free units is as follows: \\
\[ bestFit = \frac{4672503}{2147337} = 2.176 \]

\[ firstFit = \frac{4672963}{481853} = 9.70 \]
As you can see, the best-fit method is much less fragmented.

\section{Work Log}
\begin{enumerate}
\item May 30th: Concurrency 4 is finished
\item June 1st: Best-fit allocator is started
\item June 4th: Best-fit allocator is worked on
\item June 5th: Concurrency 5 is finished
\item June 8th: Best-fit allocator is finished
\end{enumerate}


\section{Version Control Log}
\begin{tabular}{l l l}
{151eb39} & tfry & Adding concurrency4.c that works! Edited makefile to compile both.\\\hline
{1c4972b} & heidiaclayton & Add missing include\\\hline
{894f1cc} & heidiaclayton & Add writeup with preliminary design\\\hline
{a4b0e63} & tfry & First pass at slob slab\\\hline
{6c316ee} & heidiaclayton & Attempt 2 at slob slab\\\hline
{a7b2a79} & heidiaclayton & Debug\\\hline
{c6b80bc} & heidiaclayton & Debug again\\\hline
{ec6aa5a} & heidiaclayton & Move struct definition\\\hline
{35b12d3} & heidiaclayton & More debugging, fix types\\\hline
{8c360f1} & heidiaclayton & Fix freeing call\\\hline
{c00ace8} & heidiaclayton & Fix cast\\\hline
{c1d80ac} & heidiaclayton & Rework, add syscall\\\hline
{425cade} & tfry & Concurrency5 added - need to make makefile\\\hline
{efabca0} & tfry & Added while loop to make it run forever\\\hline
{789a5b5} & tfry & yes\\\hline
{843ee9c} & tfry & hah\\\hline
{eb81008} & heidiaclayton & Fix slob allocator.\\\hline
{34b260b} & heidiaclayton & Move stuff around\\\hline
{f43ac32} & heidiaclayton & Bug fix\\\hline
{aedfc0c} & heidiaclayton & Move things again\\\hline
{e60e4d1} & heidiaclayton & Move things:\\\hline
{b78cb12} & tfry & Small changes\\\hline
{43269d4} & tfry & Please god work\\\hline
{d4379ca} & heidiaclayton & Revised slob\\\hline
{d3ac3e4} & heidiaclayton & Revise counters\\\hline
{47a3a1b} & heidiaclayton & Revise sys calls\\\hline
{8192b2c} & heidiaclayton & Initialize\\\hline
{d6f2470} & heidiaclayton & Add patch\\\hline
\end{tabular}

\section{What we learned}
We learned how to make our own system calls and how to test them out. We also learned a lot about the different slab allocation algorithms and the pros and cons. Best-fit has less fragmentation but can get to be very slow. Slob is faster but it causes a lot of fragmentation.

\end{document}

